\chapter{Functional programming with types}

Describing lambda calculus is important in both the context of type inference and in the context of (functional)
programming as a whole.

In this thesis we will use lambda calculus as described in Introduction to Lambda Calculus.  % TODO: insert something ABOUT LC

Lambda calculus is formally described as:

$variable ::= 'v' | variable ''' \\ \lambda\mbox{-}term ::= variable | '(' \lambda\mbox{-}term\ \lambda\mbox{-}term ')' |  '(\lambda' variable\ \lambda\mbox{-}term ')'$

% TODO: insert rules for LC

% TODO what is a sub-expression

It is a simple language in which we can formally describe any computational problem.

Choosing lambda calculus, or its derivatives might seem counterintuitive at first, because it
is not very efficient in expressing imperative ideas like mutability for example, but for
purposes of type inference we don't need to know the actual meaning of the code (like for example
if a loop stops), we can just limit ourselves to modelling type dependencies in the program and
everything on top of that is just for human readability, debugging purposes, demonstration and
primarily for possible extension.

Lambda calculus can quite easily model any C program, but one has to be careful at distinguishing
between initialization and assignment as initialization creates binding (in lambda calculus expressed
as substitution for a bound variable in an abstraction) and assignment is a function that takes two
expressions of the same type and returns the first (here we demonstrate that we don't need to
model the whole behavior of the program, because copying of the value doesn't have any effect on
the types of the arguments).

Here we will use $\lambda_C$ to denote transformation from C to lambda calculus,
this transformation can be actually done in a multiple of ways, but in this thesis we will use one that tries to
reflect the structure of the original code as closely as possible.

Models of some C constructs can be as follows (grammar taken from ): % TODO: http://www.quut.com/c/ANSI-C-grammar-y.html

$\lambda_C\left(additive\_expression '+' multiplicative\_expression\right) = \\ plus_C (\lambda_C additive\_expression) (\lambda_C multiplicative\_expression)$

This example demonstrates the most simple case where  where we can model the C construct "one to one",
but there are more tricky examples like the following one:

$\lambda_C\left(declaration\_specifiers IDENTIFIER '=' initializer\right) = \\ (\lambda IDENTIFIER : \lambda_C(declaration\_specifiers) .\ \dots) (\lambda_C initializer)$

Initialization creates binding (as stated before) so we have to model that by creating a new abstraction and put the whole
part of the function's body that follows inside this abstraction (this in effect means that all the return values of these nested functions
and of the function which contains them will share the same type), a similar thing are switch statements where the switch "call" can be modeled
as an initialization to an anonymous variable and the single case statements as assignments to this variable.

\section{Simply-Typed Lambda Calculus}

Just lambda calculus wouldn't be good enough for modeling type inference of the proposed language (We will call it CHM from now on).
% TODO: clean up this "from now on", it should be in the very top
So we need to introduce typed lambda calculi.

Simply-typed lambda calculus, described by Church, is one of the first % TODO: the very first? also: add some more info on church
lambda calculi with type system. Typing of lambda terms helps checking the validity of the program, protecting us from writing
programs that make little sense like for example adding amperes and volts % TODO: intro to lambda
and also it makes it easier to think about the program in the bigger picture giving us information about usage of the typed entities
in some code, typing $f : \alpha \rightarrow \alpha \rightarrow \alpha$, for example, tells us that $f$ is a function taking two arguments
of some type and returning a value of the same type and from that alone we can assume that applying two parameters of the same type on it
should be well-defined.

% TODO: add the stuff about variables and syntax and sh like that, also the ,\-> name and stuff, also typing

\subsection{Simply-Typed Lambda Calculus Deriving Rules}

Simply-typed lambda calculus uses the following rules:


% TODO: consistentize this with other rules...
variable:
$\dfrac{x : \tau \in \Gamma}{\Gamma \vdash x : \tau}$

application:
$\dfrac{\Gamma \vdash e : \sigma \rightarrow \tau \quad \Gamma \vdash e' : \sigma}{\Gamma \vdash e e' : \tau}$

abstraction:
$\dfrac{\Gamma, x : \sigma \vdash e : \tau}{\Gamma \vdash \lambda x . e : \sigma \rightarrow \tau}$

We call $\gamma$ either a basis, a context, or a set of assumptions. These names describe pretty well its meaning, in simply-typed lambda calculus it
is a set of typings we derive the rest of typings from.

The syntax of the rules describes step-by-step derivation of types where the bottom one follows the top one.

% TODO: check it
Example for $id = \lambda x . x$:

\begin{enumerate}
    \item $\{\} \vdash \lambda x . x : ?$
    \item $\{x : \tau\} \vdash x : \tau$ (var)
    \item $\{\} \vdash \lambda x . x : \tau \rightarrow \tau$ (abstraction) from (2), this solves (1)
\end{enumerate}

Notice here the context changes during the derivation process, so we can give typing to $\lambda\rightarrow$ terms without any context
and they can have different typings in different contexts, also one term can have multiple possible typings.

% TODO: maybe say something about church vs curry style and also unification and stuff

One thing that is a problem for the simply-typed lambda calculus is that it is not turing complete, % TODO: reference a paper that says so
because it is strongly normalizing and thus the type inference would have to solve the halting problem,
this problem will be answered by the following type systems.

\subsection{Principal Typing}

Every typeable term $e$ in simply-typed lambda calculus has principal type $\tau$ which is computable from $e$ and every other typing
$e : \sigma$ in the same context is a result of some type substitution $[\overline{\alpha} := \overline{\pi}]$, here we use the overlined symbols
as an shortcut for many atomic substitutions $[\alpha := \pi]$, where $\pi$ is a type and $\alpha$ an type variable, this will be very important
later when we introduce more advanced type systems as principal type of $e$ is the only information about $e$ we have to consider
when we see $e$ again as a subterm of some term $e'$ to check $e'$' typeability (for typeability see below). % TODO: elaborate / provide source, check and correct, be more consistent with sigmas and stuff

We can think of it as that the principal typing of $e$ is a typing general enough to be substitutable into all other possible typings of $e$
but not general enough to be substitutable into any impossible typings, this is usually simplified into "the most general type of $e$". % TODO: language and style

The example just above % TODO: maybe use fig notation
is great fit for showing this as for example $\lambda x . x : \sigma$ would be a possible typing, but not a principal one, because
$\alpha \rightarrow \alpha \rightarrow \alpha$ would give use $\alpha \sim \alpha \rightarrow \alpha$, which would give us an incomputable
infinite type.

\subsubsection{Type Instantiation}

We will call a type $\tau$ an instance of a type $\sigma$ if there is a type substitution $[\overline{\alpha} := \overline{\pi}]$
that transforms $\sigma$ into $\tau$ % TODO: do some editing

\subsection{Type Unification}

If $\tau_1, \tau_2$ are types and $S$ is a substitution, we say that $S$ unifies
$\tau_1$ and $\tau_2$ if $\tau_1 S = \tau_2 S$, then we call $S$ their unifier, and we call $\tau_1$ and $\tau_2$ unifiable. % TODO: Robinson

% TODO: maybe `simplify' the name from unifier to just unification

\subsection{Type Checking, Typeability and Type Inference}

Type checking is the process of validating whether a term can have a given type
whereas typeability is whether the term can be given some type.

Type inference is closely related to typeability but has an additional requirement of giving an actual type for the term, and furthermore
usually what we are interested in is the principal type.

Both type checking and typeability of the simply-typed lambda calculus is decidable and its type inference is computable. % TODO: add reference, check wording

\section{System F and Parametric Polymorphism} % TODO: TODO TODO TODO TODO TODO TODO TODO TODO TODO TODO TODO TODO TODO TODO TODO TODO TODO TODO TODO TODO

% TODO: add reference or I don't know
System F, polymorphic lambda calculus, or second-order lambda calculus (from the idea of quantifying types) introduces
parametric polymorphism into the lambda calculus. Here we can reuse the example of $id = \lambda x . x$ which can have many typings
$id : \tau \rightarrow \tau$ which we can state by:

$id : \forall \alpha . \alpha \rightarrow \alpha$

We call terms like $id$ beginning with $\forall$ \textbf{polymorphic} (opposed by \textbf{monomorphic}).

System F uses the rules of the simply-typed lambda calculus, plus the following:

\begin{itemize}
    \item[$\forall$-elimination] $\dfrac{\Gamma \vdash e : \forall \alpha . \sigma}{\Gamma \vdash e : \sigma [\alpha := \tau]}$
    \item[$\forall$-introduction] $\dfrac{\Gamma \vdash e : \sigma}{\Gamma \vdash e : \forall \alpha . \sigma}$
\end{itemize}

In (2) $\alpha$ is not free in any any assumptions on which the premise $e : \sigma$ depends. % TODO: introduce the terms judgement, premise, assumption

System F is still strongly normalizing and thus not turing complete. % TODO: consider rewriting/scrapping this

In system F the type checking is undecidable and thus any general type inference algorithm is impossible.
The Hindley-Milner type system is a restriction of it % TODO: elaborate on undecidability of F etc.
that allows for easy type inference algorithm we will base ours on.

\section{Hindley-Milner Type System}

The Hindley-Milner type system, we will refer to it simply as HM, is a fairly minimalistic extension of lambda calculus, or at least syntactically, however semantically it could be seen rather as a restriction of the system F which answers the undecidability of it, see the subsection Type Schemes. % TODO: doesn't have a good sound to it

Terms in HM have the form:

$e ::= x | e e' | \lambda x . e | \text{let } x = e \text{ in } e'$

(where $x$ stands for a variable, and $e, e'$ for expressions)

It differs from the syntax of lambda calculus (and that of the system F) only in the addition of the third clause, called the let statement. Meaning of this clause is the same as of $e' [x := e]$, note that this differs from a very similar construct $(\lambda x . e') e$ using abstraction instead of let in that $x$ in this latter construct cannot be polymorphic, see the next two subsections, this connects to the addition of type schemes and restriction of types, both explained later. % TODO: yeah, clarify it a little bit maybe, use shorter sentences?

Then there are two additional expressions completing the language that have no significance in the type system and thus we will not further consider them in any of the typing rules, or the inference algorithm. They are defined as follows: % TODO: wording pls

$e ::= \cdots | \text{fix } x . e | \text{if } e \text{ then } e' \text{ else } e''$
($x$ again for a variable, and $e, e', e''$ for some expression)

The latter clause with the expected meaning, and the former being defined as the least fixed point of $\lambda x . expression$. Those two clauses can be seen as any other application using the following type schemes:

$\text{fix}: \forall \alpha . ((\alpha \rightarrow \alpha) \rightarrow \alpha)$ \\
$\text{if}: \forall \alpha . \text{ bool } \rightarrow \alpha \alpha \alpha$

Note that the if clause is completely redundant in this typing system, the church booleans would be sufficient. % TODO: this is weird

% TODO: explain, also Milner 78

\subsection{Type Schemes}

We will later introduce inference rules very similar to those of the system F with some additions, which wouldn't sound that good (as general inference for the system F is impossible), however we first introduce a strict restriction on type quantification.

That restriction will be removing quantification from types and adding type schemes which replace types in assumption.

Type schemes are defined just like types in the system F. This restriction means just that the type quantification, if there is any, is always at the very front of the type scheme.

Formal definition is given as follows:

$\tau ::= \alpha | \iota | \tau \rightarrow \tau$ \\
$\sigma ::= \tau | \forall \alpha \sigma$

(where $\tau$ stands for a type, $\alpha$ a type variable, $\iota$ a primitive type (type constant), and $sigma$ a type scheme). % TODO: maybe add something to this

\subsection{Type Instantiation}

Type instantiation differs from that of system F slightly as well as explained  by... % TODO: Damas and Milner

A notable change being an addition of an \textbf{generic instance} of a type scheme $\sigma = \forall \alpha_1, \alpha_2 \cdots \alpha_n . \tau$ which is defined as $\sigma' = \forall \beta_1, \beta_2 \cdots \beta_m . \tau'$ such that
$\tau' = \tau S$ for some type substitution $S = [\alpha_1 := \tau_1, \cdots \alpha_n := \tau_n]$, where type variables $\beta_i$ are not free in $\sigma$. We then write $\sigma > \sigma'$.

Note that $\sigma > \sigma'$ implies $\sigma S > \sigma' S$, however it doesn't imply $\sigma S > \sigma$.

Note that... % FIXME: Damas and Milner use different order for substitution, but we use one which is reverse to them so we are at least a little bit more consistent.

\subsection{Inference Rules}

% TODO: instantiation
\begin{itemize}
    \item[variable] $\dfrac{}{\Gamma \vdash x : \sigma}\tiny{(x : \sigma \in \Gamma)}$
    \item[instantiation] $\dfrac{\Gamma \vdash e : \sigma}{\Gamma \vdash e : \sigma'}\tiny{(\sigma > \sigma')}$
    \item[generalization] $\dfrac{\Gamma \vdash e : \sigma}{\Gamma \vdash e : \forall \alpha . \sigma}\tiny{(\alpha \text{ not free in }\Gamma)}$
    \item[I DONT KNOW THE NAME] $\dfrac{\Gamma \vdash e : \tau' \rightarrow \tau \quad \Gamma \vdash e' : \tau' }{\Gamma \vdash e e' : \tau}$ % TODO: what is the name of this rule? It has to be application
    \item[abstraction] $\dfrac{\Gamma_x \cup \{x : \tau'\} \vdash e : \tau}{\Gamma \vdash \lambda x . e : \tau' \rightarrow \tau}$ % TODO: what is the name of this rule? It has to be application
    \item[let polymorphism] $\dfrac{\Gamma \vdash e : \sigma \quad \Gamma_x \cup \{x :\sigma\} \vdash e' : \tau}{\Gamma \vdash \text{let } x = e \text{ in } e' : \tau}$
\end{itemize}

Here $\Gamma$ stands for the context, $\sigma$ stands for a type scheme, $\tau, \tau'$ for some types, $\alpha$ for a type variable, $x$ for a regular variable, and $e, e'$ for expressions.

$\Gamma_x$ stands for the context $\Gamma$ with any assumption about $x$ removed.

The instantiation rule replaces the $\forall$-elimination rule of the system F and apart from that the let polymorphism rule is the only real extension of the rules.

Let polymorphism balances the type restriction introduced before.

\subsection{Type Inference}

We will introduce the algorithm W described by % TODO: Damas
it produces the principal type scheme for the given program (expression).
% TODO: maybe expand on it a little bit

\subsubsection{Most General Unification}

One of the main parts of the type inference algorithm is the type unification, and since we are usually after principal types, the most general unification.
We will present a variation Robinson's unification algorithm a modification of which will be used even in the next system.

Most general unifier $S$ of types $\tau_, \tau'$ (we will call it $mgu (\tau_, \tau')$) is a substitution such that for every other unifier $U$ of $\tau$ and $\tau'$  there exists a substitution $R$ such that $U = S R$. % Damas
We will further require it to use only variables from $\tau$ and $\tau'$

\subsubsection{Unification Algorithm}

% Robinson
\begin{enumerate}
    \item Set $S_0 := \epsilon$ and $k = 0$.
    \item If $\tau S_k = \tau' S_k$ then set $mgu(\tau,\tau) := S_k$ and terminate.
    \item Let $\tau_k, \tau'_k$ be leftmost well-formed sub-expressions in which $\tau$ and $\tau'$ differ (for clarification: if the expressions are represented by syntactic trees, $\tau_k$ and $\tau'_k$ have the same relative path from their respective roots - then leftmost means having the lowest inorder rank).
    \item If $\tau_k$ is a variable not occurring in $\tau'_k$, set $S_{k+1} := S_k [\tau_k := \tau'_k]$, otherwise if $\tau'_k$ is a variable not occurring in $\tau_k$, set $S_{k+1} := S_k [\tau'_k := \tau_k]$, otherwise terminate.
    \item increment $k$ by $1$ and go to step $2$.
\end{enumerate}

One of the properties of this almost naive algorithm is that it gives the most general unification and it has the desired attribute of unifying only by using the sub-expressions of $\tau, \tau'$.

\subsubsection{The Type Assignment Algorithm W}

The algorithm W is a function $W$ which takes a pair $(\Gamma, e)$ where $\Gamma$ is a set of assumptions and $e$ is an expression in the context of $\Gamma$. Then $W$ returns a pair $(S, \tau)$ such that:

$\Gamma S \vdash e : \tau$ and furthermore $close_{\Gamma S}(\tau)$ is the principal type scheme of $e$ under $\Gamma S$.

Before we describe the algorithm itself, we first need to define the closure of a type $\tau$:

$close_\Gamma(\tau) := \forall \alpha_1, \dots \alpha_n . \tau$

where $\{\alpha_1, \dots \alpha_n\} := free(\tau) \backslash free(\Gamma)$, $free$ stands for a set of free variables. % TODO: define free variables

The algorithm is described as follows:

\begin{enumerate}
    \item If $e$ is a variable and there is an assumption $e : \forall \overline{\alpha} . \tau'$ in $\Gamma$ then $S = \epsilon$ and $\tau = [\overline{\alpha} := \overline{\beta}]$ where $\overline{\beta}$ are new type variables.
    \item If $e$ matches $e_1 e_2$ then let $(S_1, \tau_1) := W(\Gamma, e_1)$, $(S_2, \tau_2) := W(\Gamma S_1, e_2)$, and $S_3 := mgu (\tau_1 S_2, \tau_2 \rightarrow \beta)$ where $\beta$ is new; then $S := S_1 S_2 S_3$ and $\tau = \beta S_3$
    \item If $e$ matches $\lambda x . e_1$ then let $\beta$ be new and $(S, \tau_1) := W(\Gamma_x \cup \{x : \beta\}, e_1)$; then $\tau = \beta S \rightarrow \tau_1$
    \item If $e$ matches $\text{let } x = e_1 \text{ in } e_2$ then let $(S_1, \tau_1) := W(\Gamma, e_1)$ and $(S_2, \tau) := W(\Gamma_x S_1 \cup \{close_{\Gamma S_1}(\tau_1)\}, e_2)$; then $S = S_1 S_2$
\end{enumerate}

If none of the cases holds, the algorithm fails as $e$ is not typeable and the algorithm can determine the reason why it isn't, that for debug purposes in a real program. % TODO: reference some paper at least

\subsection{Extensions of the Type System}

(As described in PTSfPwOaS).
Pure HM allows for just parametric polymorphism without any ad-hoc polymorphism
or subtyping, this means there is only one principal type for every symbol in the program and it has just one definition, this is useful in cases
where an algorithm (or a function) works on parameters of any types, a good example of that could be the $K$ combinator.

% FIXME: continue from here

We will break this by introducing type classes and type families, two ideas with the same basis where we put constraints on types and we can create non-overlapping instances for them, or we can put these constraints on a regular symbol to denote that the algorithm either works only on some types or doesn't make sense on others, this is especially useful in static analysis of the code.

Other solutions of this are overloading and subtyping, but the type inference of the most common form of overloading is undecidable
as described by, so overloading wouldn't be a good solution in our case. % TODO: NP overload

It also contradicts the idea of parametric polymorphism used in this thesis, where every implementation of an function can be
described as a proof of a special case of a theorem described by the principal type of the function and this type is therefore
valid for all its implementations.

Subtyping % TODO: finish this (subtyping)

\section{Overloading with Type Classes}

Type classes, as stated above, are good for stating an algorithm makes sense only for some types or that it requires a different implementation for different types, usually to enhance performance by writing specialized algorithms for common or critical types instead of writing a general slow algorithm,
this was the main motivation behind developing those, % TODO
viz different physical instructions for comparing common numeric values.

Usefulness of classes can be shown on `Ord', this constraints requires that two variables of type $\alpha$ s.t. $Ord(\alpha)$, we can say `$\alpha$ is (in) Ord'.
But not everything can be partially ordered, take directed graphs, for example, where the reachability of one vertex from another is in `Ord' only if the graph is a directed acyclic graph, in which case we can use algorithms expecting `Ord' on problems concerning reachability, but not otherwise.

`Ord' can also show how we could want different instances for different types. Let's say `Ord' has one method $<=$ of type $a -> a -> Bool$ with the expected behavior. % TODO: explain a method

For two `Int's we just compare them and return whether the first is lower than the other,
but if we have, let's say `Customer's, with different Id numbers as keys in some map (or dictionary) then we don't need to compare whole customers,
but just their Ids.

method: a function which is defined inside a class and is expected to be instantiated alongside the class. % TODO

\xxx{zhruba popsat jak to vysvihnul \citet{jones1999typing}.}

\xxx{test normalni citace: \cite{jones1999typing}.}

\section{Haskell Type System}

Haskell uses kind system % TODO: finish this (kind system)

\xxx{Teoretickej skok do budoucnosti --- mirnej review o tom jak se to ve skutecnosti dela v haskelu. Treba ukazat ze MPTCs jsou nerozhodnutelny protoze umej simulovat prolog.}
