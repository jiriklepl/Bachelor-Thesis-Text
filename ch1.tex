\chapter{Functional programming with Types}

\section{Lambda Calculus}

Understanding lambda calculus is important in both the context of type inference and in the context of (functional) programming as a whole.

In this thesis we will use lambda calculus as described in Introduction to Lambda Calculus.  % TODO: insert something ABOUT LC


\begin{defn}[Lambda Calculus Grammar]
$$variable ::= v\ |\ variable'$$
$$\lambda\mbox{-}term ::= variable\ |\ (\lambda\mbox{-}term\ \lambda\mbox{-}term)\ |\ (\lambda\ variable\ . \lambda\mbox{-}term )$$
\end{defn}

In the $\lambda\mbox{-}term$ grammar rule, the second case is called \emph{application} and the third is called \emph{abstraction}. These two constructs correspond to passing and declaring an argument respectively.

It is a simple language in which we can formally describe any computational problem.

The whole idea of evaluation of a lambda calculus program is using a simple rewriting rule, $\beta$ reduction:

\begin{defn}[$\beta$ reduction]
    $$(\lambda x . e) e' = e [x := e']$$ % TODO: Barendregt reference
\end{defn}

There is one other rewriting rule, \emph{$\alpha$ conversion}, which is used to resolve name collisions, but we will not consider it as we will consider each bound variable uniquely named.

\subsection{Subexpression}

\begin{defn}[Subexpression]
Given a well-defined expression $e$, a subexpression is an expression $e'$ such that: $e = \lambda x . e'$, $e = e' e''$, $e = e'' e'$, or $e = e'$, and iteratively from that.
\end{defn}

This definition will extend intuitively to all other syntactic systems.

\section{Simply-Typed Lambda Calculus}

Simply-typed lambda calculus, described by Church, is one of the first % TODO: the very first? also: add some more info on church
lambda calculi with a type system. Typing of lambda terms helps checking the validity of the program, protecting us from writing
programs that make little sense like for example adding amperes and volts % TODO: intro to lambda
and also it makes it easier to think about the program in the bigger picture giving us information about the usage of the typed entities
in some code, typing $f : \alpha \rightarrow \alpha \rightarrow \alpha$, for example, tells us that $f$ is a function taking two arguments
of some type and returning a value of the same type and from that alone we can assume that applying two parameters of the same type on it
should be well-defined.

% TODO: add the stuff about variables and syntax and sh like that, also the ,\-> name and stuff, also typing

\subsection{Simply-Typed Lambda Calculus Deriving Rules}

Simply-typed lambda calculus uses the following rules:

\begin{defn}[Inference rules]
    $$\boxed{\begin{array}{l c}
        \infer{\Gamma \vdash x : \tau}{x : \tau \in \Gamma} & (\text{var}) \\\\
        \infer{\Gamma \vdash e e' : \tau}{\Gamma \vdash e : \sigma \rightarrow \tau \quad \Gamma \vdash e' : \sigma} & (\text{application}) \\\\
        \infer{\Gamma \vdash \lambda x . e : \sigma \rightarrow \tau}{\Gamma, x : \sigma \vdash e : \tau} & (\text{abstraction})
    \end{array}}$$
\end{defn}

We call $\Gamma$ either a basis, a context, or a set of assumptions. These names describe pretty well its meaning, in the simply-typed lambda calculus it is a set of typings we derive the rest of typings from.

The syntax of the rules describes the step-by-step derivation of types where the bottom one follows the top one. % TODO: reword

% TODO: check it
\subsubsection{Example}
A Step by step example for $id = \lambda x . x$:

\begin{enumerate}
    \item $\{\} \vdash \lambda x . x : ?$
    \item $\{x : \tau\} \vdash x : \tau$ (var)
    \item $\{\} \vdash \lambda x . x : \tau \rightarrow \tau$ (abstraction) from (2), this solves (1)
\end{enumerate}

Notice here the context changes during the derivation process, so we can give typing to $\lambda\rightarrow$ terms without any context and they can have different typings in different contexts, also one term can have multiple possible typings.

% TODO: maybe say something about church vs curry style and also unification and stuff

One thing that is a problem for the simply-typed lambda calculus is that it is not turing complete, % TODO: reference a paper that says so
because it is strongly normalizing and thus the type inference would have to solve the halting problem, this problem will be answered by the following type systems.

\subsection{Principal Typing}

Every typeable term $e$ in simply-typed lambda calculus has principal type $\tau$ which is computable from $e$ and every other typing $e : \sigma$ in the same context is a result of some type substitution $[\overline{\alpha} := \overline{\pi}]$, here we use the overlined symbols as a shortcut for many atomic substitutions $[\alpha := \pi]$, where $\pi$ is a type and $\alpha$ a type variable, this will be very important later when we introduce more advanced type systems as the principal type of $e$ is the only information about $e$ we have to consider when we see $e$ again as a subterm of some term $e'$ to check $e'$' typeability (for typeability see below). % TODO: elaborate / provide source, check and correct, be more consistent with sigmas and stuff

We can think of it as that the principal typing of $e$ is a typing general enough to be substitutable into all other possible typings of $e$
but not general enough to be substitutable into any impossible typings, this is usually simplified into "the most general type of $e$". % TODO: language and style

The example just above % TODO: maybe use fig notation
is great fit for showing this as for example $\lambda x . x : \sigma$ would be a possible typing, but not a principal one, because
$\alpha \rightarrow \alpha \rightarrow \alpha$ would give use $\alpha \sim \alpha \rightarrow \alpha$, which would give us an incomputable
infinite type.

\subsubsection{Type Instantiation}

We will call a type $\tau$ an instance of a type $\sigma$ if there is a type substitution $[\overline{\alpha} := \overline{\pi}]$
that transforms $\sigma$ into $\tau$ % TODO: do some editing

\subsection{Type Unification}

If $\tau_1, \tau_2$ are types and $S$ is a substitution, we say that $S$ unifies $\tau_1$ and $\tau_2$ if $\tau_1 S = \tau_2 S$, then we call $S$ their unifier, and we call $\tau_1$ and $\tau_2$ unifiable. % TODO: Robinson

% TODO: maybe `simplify' the name from unifier to just unification

\subsection{Type Checking, Typeability and Type Inference}

Type checking is the process of validating whether a term can have a given type whereas typeability is whether the term can be given some type. % TODO: reword this!

Type inference is closely related to typeability but has an additional requirement of giving an actual type for the term, and furthermore usually what we are interested in is the principal type.

Both type checking and typeability of the simply-typed lambda calculus is decidable and its type inference is computable. % TODO: add reference, check wording

\section{System F and Parametric Polymorphism} % TODO: TODO TODO TODO TODO TODO TODO TODO TODO TODO TODO TODO TODO TODO TODO TODO TODO TODO TODO TODO TODO

% TODO: add reference or I don't know
System F, polymorphic lambda calculus, or second-order lambda calculus (from the idea of quantifying types) introduces parametric polymorphism into the lambda calculus. Here we can reuse the example of $id = \lambda x . x$ which can have many typings $id : \tau \rightarrow \tau$ which we can state by:

$id : \forall \alpha . \alpha \rightarrow \alpha$

We call terms like $id$ beginning with $\forall$ \textbf{polymorphic} (opposed by \textbf{monomorphic}).

\begin{defn}[Inference rules]
    System F uses the rules of the simply-typed lambda calculus, plus the following:
    $$\boxed{\begin{array}{l c}
        \infer{\Gamma \vdash e : \sigma [\alpha := \tau]}{\Gamma \vdash e : \forall \alpha . \sigma} & (\text{$\forall$-elimination})\\\\
        \infer{\Gamma \vdash e : \forall \alpha . \sigma}{\Gamma \vdash e : \sigma} & (\text{$\forall$-introduction})
    \end{array}}$$
    In $(\text{$\forall$-introduction})$ the type variable $\alpha$ is not free in any any assumptions on which the premise $e : \sigma$ depends. % TODO: introduce the terms judgement, premise, assumption
\end{defn}

In system F the type checking is undecidable and thus any general type inference algorithm is impossible.

The Hindley-Milner type system is a restriction of it % TODO: elaborate on the undecidability of F etc.
that allows for easy type inference algorithm we will base ours on.

\section{Hindley-Milner Type System}

The Hindley-Milner type system, we will refer to it simply as HM, is a fairly minimalistic extension of lambda calculus, or at least syntactically, however semantically it could be seen rather as a restriction of the system F which answers the undecidability of it, see the subsection Type Schemes. % TODO: doesn't have a good sound to it

Terms in HM have the form:

$$e ::= x\ |\ e e'\ |\ \lambda x . e\ |\ \text{let } x = e \text{ in } e'$$ % TODO: defn

(where $x$ stands for a variable, and $e, e'$ for expressions)

It differs from the syntax of lambda calculus (and that of the system F) only in the addition of the third clause, called the let statement. Meaning of this clause is the same as of $e' [x := e]$, note that this differs from a very similar construct $(\lambda x . e') e$ using abstraction instead of let in that $x$ in this latter construct cannot be polymorphic, see the next two subsections, this connects to the addition of type schemes and restriction of types, both explained later. % TODO: yeah, clarify it a little bit maybe, use shorter sentences?

Then there are two additional expressions completing the language that have no significance in the type system and thus we will not further consider them in any of the typing rules, or the inference algorithm. They are defined as follows: % TODO: wording pls

$e ::= \cdots | \text{fix } x . e | \text{if } e \text{ then } e' \text{ else } e''$ % TODO: defn
($x$ again for a variable, and $e, e', e''$ for some expression)

The latter clause with the expected meaning, and the former being defined as the least fixed point of $\lambda x . expression$. Those two clauses can be seen as any other application using the following type schemes:

$\text{fix}: \forall \alpha . ((\alpha \rightarrow \alpha) \rightarrow \alpha)$ \\
$\text{if}: \forall \alpha . \text{ bool } \rightarrow \alpha \alpha \alpha$ % TODO: FORMAT

Note that the `if' clause is completely redundant in this typing system, the church booleans would be sufficient. % TODO: this is weird

% TODO: explain, also Milner 78

\subsection{Types}

Types of the HM type system are restricted in that they cannot include quantification of type variables introduced in the F system.

\begin{defn}[Type]
    $$\tau ::= \alpha\ |\ \iota\ |\ \tau \rightarrow \tau$$
    where $\tau$ stands for a type, $\alpha$ for a type variable, $\iota$ for a primitive type (type constant), and $sigma$ for a type scheme. % TODO: maybe add something to this
\end{defn}

\subsection{Type Schemes}

We will later introduce inference rules very similar to those of the system F with some additions, which would not sound that good (as general inference for the system F is impossible). % FIXME: sound, move below

Type schemes are defined just like types in the system F. This restriction means just that the type quantification, if there is any, is always at the very front of the type scheme.

\begin{defn}[Type Scheme]
    $$\sigma ::= \tau\ |\ \forall \alpha . \tau$$
    where the former case is a monomorphic type scheme and the latter is polymorphic
\end{defn}

\subsubsection{Typing extensions}

In the HM type system it is common to introduce the following type operators:

Cartesian product (tuples): $\times$; disjoint sum (tagged unions): $+$; and lists. Existence of these is not required for the type system and so we will avoid them going forward, their reintroduction is trivial.

\subsection{Type Instantiation}

Type instantiation differs from that of system F slightly as well as explained  by... % TODO: Damas and Milner

A notable change being an addition of an \textbf{generic instance} of a type scheme $\sigma = \forall \alpha_1, \alpha_2 \cdots \alpha_n . \tau$ which is defined as $\sigma' = \forall \beta_1, \beta_2 \cdots \beta_m . \tau'$ such that $\tau' = \tau S$ for some type substitution $S = [\alpha_1 := \tau_1, \cdots \alpha_n := \tau_n]$, where type variables $\beta_i$ are not free in $\sigma$. We then write $\sigma > \sigma'$.

Note that $\sigma > \sigma'$ implies $\sigma S > \sigma' S$, however it doesn't imply $\sigma S > \sigma$.

Note that... % FIXME: Damas and Milner use a different order for substitution, but we use one which is reverse to them so we are at least a little bit more consistent.

\subsection{Inference Rules}

% TODO: instantiation
\begin{defn}[Inference rules]
    $$\boxed{\begin{array}{l c}
        \infer[(x : \sigma \in \Gamma)]{\Gamma \vdash x : \sigma}{} & (\text{var}) \\\\
        \infer{\Gamma \vdash e e' : \tau}{\Gamma \vdash e : \tau' \rightarrow \tau \quad \Gamma \vdash e' : \tau'} & (\text{application}) \\\\
        \infer{\Gamma \vdash \lambda x . e : \tau' \rightarrow \tau}{\Gamma_x \cup \{x : \tau'\} \vdash e : \tau} & (\text{abstraction}) \\\\
        \infer[(\sigma > \sigma')]{\Gamma \vdash e : \sigma'}{\Gamma \vdash e : \sigma} & (\text{instantiation}) \\\\
        \infer[(\alpha \text{ not free in }\Gamma)]{\Gamma \vdash e : \forall \alpha . \sigma}{\Gamma \vdash e : \sigma} & (\text{generalization}) \\\\
        \infer{\Gamma \vdash \text{let } x = e \text{ in } e' : \tau}{\Gamma \vdash e : \sigma \quad \Gamma_x \cup \{x :\sigma\} \vdash e' : \tau} & (\text{let polymorphism})
    \end{array}}$$
    Here $\Gamma$ stands for the context, $\sigma$ stands for a type scheme, $\tau, \tau'$ for some types, $\alpha$ for a type variable, $x$ for a regular variable, and $e, e'$ for expressions. $\Gamma_x$ stands for the context $\Gamma$ with any assumption about $x$ removed.
\end{defn}

The instantiation rule replaces the $\forall$-elimination rule of the system F and apart from that the let polymorphism rule is the only real extension of the rules, it balances the type restriction introduced before.

\subsection{Type Inference}

We will introduce the algorithm W described by % TODO: Damas
it produces the principal type scheme for the given program (expression).
% TODO: maybe expand on it a little bit

\subsubsection{Most General Unification}

\begin{algorithm}[t]
\caption{Unification Algorithm}
\label{mgu}
\begin{algorithmic}[1]
\Function{$mgu$}{$\tau, \tau'$}
    \State $S_0 \gets \epsilon$
    \For{$k = 0$ to $\infty$}
    \If{$\tau S_k = \tau' S_k$}
        \Return $S_k$
    \EndIf
    \State Let $\tau_k, \tau'_k$ be leftmost corresponding well-formed sub-expressions in which $\tau$ and $\tau'$ differ \Comment if the expressions are represented by syntactic trees, $\tau_k$ and $\tau'_k$ have the same relative path from their respective roots - then leftmost means having the lowest inorder rank).
    \If{$\tau_k$ is a variable not occurring in $\tau'_k$}
        \State $S_{k+1} \gets S_k [\tau_k := \tau'_k]$
    \ElsIf{$\tau'_k$ is a variable not occurring in $\tau_k$}
        \State $S_{k+1} \gets S_k [\tau'_k := \tau_k]$
    \Else
        \State Terminate with an \textit{occurrence check} error
    \EndIf
    \EndFor
\EndFunction
\end{algorithmic}
\end{algorithm}

One of the main parts of the type inference algorithm is the type unification, and since we are usually after principal types, the most general unification.

The most general unifier of types $\tau_, \tau'$ is a unifier $S$ such that for every other unifier $U$ of $\tau$ and $\tau'$ there exists a substitution $R$ such that $U = S R$. % TODO: Damas, also maybe move to the definition of unifier

We will further require the algorithm to use only variables from $\tau$ and $\tau'$.

We present an algorithm \ref{mgu} which is a variation of Robinson's unification algorithm. We will use a modification of this algorithm even in the next type system. % TODO: Robinson

\subsubsection{The Type Assignment Algorithm W}

\begin{algorithm}[t]
\caption{The algorithm W}
\label{w}
\begin{algorithmic}[1]
\Function{W}{$\Gamma, e$}
    \If{$e$ is a variable $\wedge\ e : \forall \overline{\alpha} . \tau' \in \Gamma$}
    \State let $\overline{\beta}$ be a list of \emph{new} type variables s.t. $|\overline{\beta}| = |\overline{\alpha}|$
    \State $(S, \tau) \gets (\epsilon, \tau' [\overline{\alpha} := \overline{\beta}])$
    \ElsIf{$e$ \textbf{matches} $e_1 e_2$}
    \State $(S_1, \tau_1) \gets W(\Gamma, e_1)$
    \State $(S_2, \tau_2) \gets W(\Gamma S_1, e_2)$
    \State let $\beta$ be a \emph{new} type variable
    \State $S_3 \gets mgu (\tau_1 S_2, \tau_2 \rightarrow \beta)$
    \State $(S, \tau) \gets (S_1 S_2 S_3, \beta S_3)$
    \ElsIf{$e$ \textbf{matches} $\lambda x . e_1$}
    \State let $\beta$ be \emph{new} type variable
    \State $(S, \tau_1) \gets W(\Gamma_x \cup \{x : \beta\}, e_1)$
    \State $\tau \gets (\beta S \rightarrow \tau_1)$
    \ElsIf{$e$ \textbf{matches} $\text{let } x = e_1 \text{ in } e_2$}
    \State $(S_1, \tau_1) \gets W(\Gamma, e_1)$
    \State $(S_2, \tau) \gets W(\Gamma_x S_1 \cup \{close_{\Gamma S_1}(\tau_1)\}, e_2)$
    \State $S = S_1 S_2$
    \Else
    \State Terminate with an error stating that $e$ is not typeable
    \EndIf
    \State \Return $(S, \tau)$
\EndFunction
\end{algorithmic}
\end{algorithm}

Before we describe the algorithm itself, we first need to define the closure of a type $\tau$ under assumptions $\Gamma$:

\begin{defn}[$close_\Gamma(\tau)$]
    \label{defn:close}
    $$close_\Gamma(\tau) ::= \forall \alpha_1, \dots \alpha_n . \tau$$
    where $\{\alpha_1, \dots \alpha_n\} := free(\tau) \backslash free(\Gamma)$, $free$ stands for the set of free variables. % TODO: define free variables
\end{defn}

The algorithm \ref{w} is a function $W$ which takes a pair $(\Gamma, e)$ where $\Gamma$ is a set of assumptions and $e$ is an expression in the context of $\Gamma$. Then $W$ returns a pair $(S, \tau)$ such that:

$\Gamma S \vdash e : \tau$ and furthermore $close_{\Gamma S}(\tau)$ is a principal type scheme of $e$ under $\Gamma S$.

\subsection{Extensions of the Type System}

(As described in PTSfPwOaS).
Pure HM allows for just parametric polymorphism. However without any ad-hoc polymorphism or subtyping, this means there is only one principal type for every symbol in the program and it has just one definition, this is useful in cases where an algorithm (or a function) works on parameters of any types, a good example of that could be the $K$ combinator (sometimes called `const function'). We call these algorithms generic.

The problem is some functions work only on a particular type, for example basic arithmetic functions. We would like to be able to say, that it makes sense to add two integers, or two real numbers, but not adding two boolean values.

A trivial extension would be the introduction of tagged unions % TODO: see Type schemes
But if we were to define the $+$ operator as: $(+) : Int + Real \rightarrow Int + Real \rightarrow Int + Real$ we solve the problem with booleans, one different problem comes to the surface: we would like to say that adding two integers yields an integer whereas adding two real numbers yields a real number, one more argument could also be that we would like to forbid adding integers to real numbers (maybe we want to require all data conversions be explicit).

So what we would like is some way of expressing that $(+)$ takes two parameters of a type we can perform addition on and that the function yields us a result of the same type.

And an answer to this can be the introduction of \textbf{type classes}.
% TODO maybe: note that they are combinable with tagged unions for better expressiveness

\subsubsection{Type classes}

The idea of type classes (parametrized by generic types) is that we put constraints on types of some variables (or functions) we use in our algorithms. Those constraints tell us that the usage of these algorithms is valid only in the context where the constraints are satisfied (this allows us to implement otherwise impossible generic algorithms, for example sorting).

And also the functions defined inside a type class (we will call them methods) can have a specialized implementation for each \emph{instance}, this was actually the main motivation for their creation: $Eq$ class with a method $==$ which needs to be specialized for each type. % TODO: reference something

\textbf{Instances} of type classes (parametrized by a type instance of the type of the corresponding type class) are what allows for ad-hoc polymorphism % TODO: add some mention of ad-hoc poly more above
in a very well controlled matter, they are a limited (but powerful) version of classic overloading where all instances have to share a common supertype and they cannot overlap (share a common type instance - this can be proven by the unification algorithm).

Type classes can be added to the algorithm $W$ defined above with only little changes. % TODO: reference something, fix wording

\subsubsection{Subtyping and Overloading}

% FIXME: this is just weird

Subtyping and overloading are slightly different approaches to the same problem we discussed with specifying what types the $(+)$ function can act on.

If we want to use either of those we have to constrain them somehow or they make type inference undecidable. We will use the variant defined by % TODO Smith

Subtyping however doesn't solve that problem very well as if we specify that $Int$ is a subtype of $Real$ then we have to accept that $(+)$ applied to two real operands can still return an integer. That doesn't seem that bad from the mathematical point of view, but it can lead to implicit data conversions which can be undesirable, especially if we were to use the type system in a context of a low-level programming. % FIXME

Overloading constrained similarly as before in this context then makes these two approaches very similarly expressive. % TODO: check it, also

For the overloading it is not only undecidable but it also contradicts the ideas of parametric polymorphism used in this thesis, where every implementation of an function can be described as a proof of a special case of a theorem described by the principal type of the function and this type is therefore valid for all its implementations. % FIXME: doesn't sound right

\section{Overloading with Type Classes}

Type classes, as stated before, are good for stating an algorithm makes sense only for some types or that it requires a different implementation for different types, usually to enhance performance by writing specialized algorithms for common or critical types instead of writing a general slow algorithm, this was the main motivation behind developing those, % TODO: sounds like a contradiction
viz different physical instructions for comparing common numeric values.

Usefulness of classes can be shown on `Ord', this constraints requires that two variables of type $\alpha$ s.t. $Ord(\alpha)$, we can say `$\alpha$ is (in) Ord'. But not everything can be partially ordered, take directed graphs, for example, where the reachability of one vertex from another is in `Ord' only if the graph is a directed acyclic graph, in which case we can use algorithms expecting `Ord' on problems concerning reachability, but not otherwise.

`Ord' can also show how we could want different instances for different types. Let's say `Ord' has one method $<=$ of type $a \rightarrow a \rightarrow Bool$ with the expected behavior. % TODO: explain a method

For two `Int's we just compare them and return whether the first is lower than the other, but if we have, let's say `Customer's, with different Id numbers as keys in some map (or dictionary) then we don't need to compare whole customers, but just their Ids.

\section{Haskell Type system}

For our purposes we will (limit) the haskell language significantly, ignoring any pattern-matching features and almost all syntax sugar, the only important feature for us being the type system. % TODO: find better words for this

\subsection{Type Constructors}

The haskell generalizes the HM type system and the HM syntax mostly via the notion of type constructors. If they take no type arguments then we call them nullary, those are the ones we already call \emph{types} in the HM type system. We call a type constructor n-ary when it takes exactly n type arguments, for example the type constructor $(\rightarrow)$ is binary and constructs the type of a function. % TODO: rewrite it slightly

We can formally define type constructors similarly to functions.

\subsection{Kinds}

The next thing the haskell type system adds to the HM type system is the notion of kinds, kinds are used for classification of type constructors and for checking the validity of their usage in the program.

Kinds reflect the nature of functions and they follow their syntactic rules, but applied to type constructors instead of values. The most simple kind $*$ (star) represents all types capable of having a value: all legal types from the classic HM type system fall to this category.

Then we have kinds of a form $k_1 \rightarrow k_2$ which take a type of a kind $k_1$ and return a type of kind $k_2$. For example a $(\rightarrow)$ type constructor has a kind $* \rightarrow (* \rightarrow *)$, usually written with considering right associativity as $* \rightarrow * \rightarrow *$. This means $(\rightarrow)$ takes two types and returns a new type - this reflects the fact that a type of a function is defined by its parameter's type and its return type.

Kinds are less complex then types as they are defined recursively only from the two rules defined in the previous two paragraphs.

For example we can define a very common type constructor $List: * \rightarrow *$ which takes a type $a$ and returns a type representing a linked list of values of type $a$. We can use this type constructor to show the reason behind the notion of kinds as it wouldn't make much sense for a value to have a type just $List$ without it being applied to any type. % TODO: fix the last part (and possibly the whole subsection)

We will formally define kinds as:

\begin{defn}[kinds]
    $$k ::= *\ |\ k' \rightarrow k''$$
    where $k, k', k''$ are kinds.
\end{defn}

\subsection{Types}

We will define types formally as:

\begin{defn}[types]
    $$\tau ::= x\ |\ c : k\ |\ \tau \tau'$$
    where $\tau, \tau'$ are types, $x$ is a type variable, $c$ is a type constant, and $k$ is a kind. We will, for example consider the $(\rightarrow)$ type constructor a regular type of the kind $* \rightarrow * \rightarrow *$.
\end{defn}

\subsection{Type Schemes}

Type schemes will be defined similarly to those of the HM type system, differing only in what we consider a type and in that there can be constraints that the types have to satisfy. It should be also noted that only types of the $*$ kind can be assigned to values.

\begin{defn}[Type Schemes]
    $$\sigma ::= \forall \alpha_1, \dots \alpha_n . qt\ |\ qt$$
    where $qt$ stands for a qualified type (a constrained type).
\end{defn}

\subsubsection{Qualified Types and Predicates}

A qualified type is a type with an added set of predicates every instance of it has to satisfy.

\begin{defn}[Qualified Types and Predicates]
    $$qt ::= \{\beta_1 \in C_1, \beta_2 \in C_2, \dots \beta_m \in C_m\} \Rightarrow \tau\ |\ \tau$$
    where $\alpha_i, \beta_j$ name type variables and $\{\beta_1, \dots \beta_m\} \subseteq \{\alpha_1, \dots \alpha_n\}$, $C_j$ names a type class, and $\tau$ names a type. We call the pair $\beta_j \in C_j$ a \textbf{predicate (constraint)} % TODO: check those betas
\end{defn}

Going forward we will assume $\forall \{\}. qt = qt$ and $\{\} \Rightarrow \tau = \tau$ for simplifying other definitions, this requires only a trivial transformation. % TODO: fix this

\subsection{Type substitution}

With the introduction of qualified types, we should explicitly state that the type substitution applies to qualified types just like as if the type parameters of the predicates were a part of the type itself.

\subsection{Type Inference Rules}

The type inference rules for the haskell type system are almost the same as the type inference rules for the HM type system, but we have to add rules for the inference of the type class constraints:

\begin{defn}[Inference rules]
    $$\boxed{\begin{array}{l c}
        \infer[(x : \sigma \in \Gamma)]{\Gamma \vdash x : \sigma}{} & (\text{var}) \\\\
        \infer{\Gamma \vdash e e' : \mathbf{P} \cup \mathbf{P'}  \Rightarrow \tau}{\Gamma \vdash e : \mathbf{P} \Rightarrow \tau' \rightarrow \tau & \Gamma \vdash e' : \mathbf{P'} \Rightarrow \tau'} & (\text{application}) \\\\
        \infer{\Gamma \vdash \lambda x . e : \mathbf{P} \cup \mathbf{P'} \Rightarrow \tau' \rightarrow \tau}{\Gamma_x \cup \{x : \mathbf{P'} \Rightarrow \tau'\} \vdash e : \mathbf{P} \Rightarrow \tau} & (\text{abstraction}) \\\\
        \infer[(\sigma > \sigma')]{\Gamma \vdash e : \sigma'}{\Gamma \vdash e : \sigma} & (\text{instantiation}) \\\\
        \infer[(\alpha \text{ not free in }\Gamma)]{\Gamma \vdash e : \forall \alpha . \sigma}{\Gamma \vdash e : \sigma} & (\text{generalization}) \\\\
        \infer{\Gamma \vdash \text{let } x = e \text{ in } e' : \mathbf{P} \cup \mathbf{P'} \Rightarrow \tau'}{\Gamma \vdash e : \forall \bar{\alpha} .  \mathbf{P} \Rightarrow \tau & \Gamma_x \cup \{x :\forall \bar{\alpha} .  \mathbf{P} \Rightarrow \tau \} \vdash e' : \mathbf{P'} \Rightarrow \tau'} & (\text{let polymorphism})
    \end{array}}$$
    where $\tau, \tau'$ stand for types, $e, e'$ stand for expressions, $x$ stands for a type variable, $\bar{\alpha}$ stands for a list of type variables, $\Gamma$ stands for the set of assumptions, $\Gamma_x$ stands for the set of assumptions without any judgement of x, and $\mathbf{P}, \mathbf{P'}$ stand for type predicates.
\end{defn}

Those rules are the same as those of the HM type system with only a trivial difference that complex expressions retain predicates of all of their subexpressions.


\subsection{Type Instances and Type Unification Algorithm}

% FIXME: yeah there is some difference to the HM type system here... we have to check and deduce kinds

The addition of type constructors and type classes does not require any change to the definition of the type unification algorithm.

As for type instances, the definition is without any change.

\subsection{The Type Inference Algorithm}

% FIXME: FROM MIREK zhruba popsat jak to vysvihnul \citet{jones1999typing

% TODO: test normalni citace: \cite{jones1999typing}.

% TODO: FROM MIREK Teoretickej skok do budoucnosti --- mirnej review o tom jak se to ve skutecnosti dela v haskelu. Treba ukazat ze MPTCs jsou nerozhodnutelny protoze umej simulovat prolog.

% TODO: FROM MIREK zhruba popsat jak to vysvihnul \citet{jones1999typing}.

% TODO: FROM MIREK test normalni citace: \cite{jones1999typing}.
