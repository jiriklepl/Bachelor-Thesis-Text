\chapter{Functional programming with Types}

\section{Lambda Calculus}

Understanding lambda calculus is important in both the context of type inference and in the context of (functional) programming.

In this thesis, we will use lambda calculus as described in Introduction to Lambda Calculus.

\begin{defn}[Lambda Calculus Grammar]
$$variable ::= v\ |\ variable'$$
$$\lambda\mbox{-}term ::= variable\ |\ (\lambda\mbox{-}term\ \lambda\mbox{-}term)\ |\ (\lambda\ variable\ . \lambda\mbox{-}term )$$
\end{defn}

In the $\lambda\mbox{-}term$ grammar rule, the second case is called \emph{application} and the third is called \emph{abstraction}. These two constructs correspond to passing and declaring an argument, respectively.

\begin{defn}[Bound Variable and Free Variable]
    \label{defn:boundFree}
    In the abstraction we call the variable $x$ (its occurrences in the abstraction) \textbf{bound} by the abstraction. If it is not bound we call it \textbf{free}.
\end{defn}

It is a simple language in which we can formally describe any computational problem.

The whole idea of evaluation of a lambda calculus program is using a simple rewriting rule, $\beta$ reduction:

\begin{defn}[$\beta$ reduction]
    $$(\lambda x . e) e' = e [x := e']$$ \cite{barendregt1992lambda}
\end{defn}

There is one other rewriting rule, \emph{$\alpha$ conversion}, which is used to resolve name collisions, but we will not consider it as we will consider each bound variable uniquely named.

\subsection{Subexpression}

\begin{defn}[Subexpression]
Given a well-defined expression $e$, a subexpression is an expression $e'$ such that: $e = \lambda x . e'$, $e = e' e''$, $e = e'' e'$, or $e = e'$, and iteratively from that.
\end{defn}

This definition will extend intuitively to all other syntactic systems.

\subsection{Writing Conventions}

Lambda calculi usually use the writing convention of applications being left-associative and abstractions being right-associative


\section{Simply-Typed Lambda Calculus }

Simply-typed lambda calculus (also $\lambda\rightarrow$), described by Church, is one of the first lambda calculi with a type system. Typing of lambda terms helps in checking the validity of the program, protecting the user from writing programs that make little sense like for example, adding amperes to volts \cite{barendregt1992lambda} and also it makes it easier to think about the program in the bigger picture giving us information about the usage of the typed entities in some code, typing $f : \alpha \rightarrow \alpha \rightarrow \alpha$, for example, tells us that $f$ is a function taking two arguments of some type and returning a value of the same type. From that alone, we can assume that applying it to two parameters of the same type should be well-defined.

Syntactically it is not any different from the lambda calculus, considering the \emph{\`a la} Curry variant. \cite{barendregt1992lambda}

\subsection{Type}

We will formally define what we consider a type, first we consider a \textbf{type variable} similarly to the variables of the lambda calculus:

\begin{defn}[Type Variable]
    $$type\_variable ::= \alpha\ |\ type\_variable'$$
\end{defn}

Then the types can be formally defined:

\begin{defn}[Type]
    \label{defn:typeSTLC}
    $$\tau ::= \alpha\ |\ \tau \rightarrow \tau'$$
    where $\tau, \tau'$ stand for types and $\alpha$ for a type variable. $\tau \rightarrow \tau'$ is a type of a function taking an argument of a type $\tau$ and returning a value of a type $\tau'$.
\end{defn}

We can also freely add type constants (like \emph{integers} or \emph{booleans}) to the definition.

\subsection{Typing}

We will use the formal definition for typing as follows:

\begin{defn}[Typing and Assumption]
    \label{defn:typingSTLC}
    $$e : \tau$$
    where $e$ is an expression and $\tau$ is a type. If $e$ is a variable $x$ and we infer other types from it, we refer to such typing as an \textbf{assumption} or a \textbf{judgement}.
\end{defn}

Sometimes we will write $\Gamma \vdash e : \tau$, meaning that in the context of $\Gamma$ (set of assumptions) $e$ can get the type $\tau$, more on that in the definition \ref{defn:STLCInference}.

\subsection{Inference Rules}

Simply-typed lambda calculus uses the following rules, we present them to formally define how to determine whether a certain typing is valid:

\begin{defn}[Inference rules]
    \label{defn:STLCInference}
    $$\boxed{\begin{array}{l c}
        \infer[(x : \tau \in \Gamma)]{\Gamma \vdash x : \tau}{} & (\text{var}) \\\\
        \infer{\Gamma \vdash e e' : \tau}{\Gamma \vdash e : \sigma \rightarrow \tau \quad \Gamma \vdash e' : \sigma} & (\text{application}) \\\\
        \infer{\Gamma \vdash \lambda x . e : \sigma \rightarrow \tau}{\Gamma, x : \sigma \vdash e : \tau} & (\text{abstraction})
    \end{array}}$$
\end{defn}

We call $\Gamma$ either a basis, a context, or a set of assumptions. These names describe pretty well its meaning. In the simply-typed lambda calculus, it is a set of typings of type variables from which we derive the rest of the typings.

The inference rules describe the step-by-step derivation of types where the bottom one follows the top one.

\subsubsection{Example}
A Step by step example for $id = \lambda x . x$:

\begin{enumerate}
    \item $\{\} \vdash \lambda x . x : ?$
    \item $\{\} \vdash \lambda x . x : \sigma \rightarrow \tau$ (abstraction) \label{enum:abs}
    \item $\{\}, x : \sigma \vdash x : \tau$ (still abstraction; we want to infer $\tau$)
    \item $\{x : \sigma\} \vdash x : \sigma$ (var)
    \item $\{\} \vdash \lambda x . x : \sigma \rightarrow \sigma$ (result of abstraction in the step \ref{enum:abs})
\end{enumerate}

Notice here the context changes during the derivation process and that we can give typing to $\lambda\rightarrow$ terms without any context, and they can have different typings in different contexts. Also, one term can have multiple possible typings.

% TODO: maybe say something about Church vs curry style

\subsection{Type substitution}

Type substitution is a process that rewrites (simultaneously) certain type variables into different types. Substitutions will be very important for various definitions and, even more importantly, for type unification, which will be the basis for type inference algorithms for the following type systems.

\begin{defn}[Type Substitution]
    \label{defn:substitution}
    $$S ::= [\alpha_1 := \pi_1, \dots \alpha_n := \pi_n]$$
    here $S$ stands for a type substitution, each $\alpha_i$ stands for a distinct type variable and $\pi_i$ for a type.
\end{defn}

We will write applying $S$ to a type $\tau$ as $\tau S$, we will also allow a simplified definition $[\overline{\alpha} := \overline{\pi}]$ meaning $[\alpha_1 := \pi_1, \dots \alpha_n := \pi_n]$.

\subsection{Principal Typing}

Every typeable term $e$ in the simply-typed lambda calculus has principal type $\tau$ \cite{barendregt1992lambda} which is computable from $e$ and for every other valid typing $e : \sigma$ in the same context there is a substitution $S$ such that $\sigma = \tau S$.

The intuitive idea of principal typing of $e$ is that it is a typing general enough to be substitutable into all other possible typings of $e$ but not general enough to be substitutable into any impossible typings. This is usually simplified into `the most general type of $e$'.

The example just above is a great fit for showing this as for example $\lambda x . x : \sigma$ would be a possible typing, but not a principal one, because substitution $[\sigma := \alpha \rightarrow \alpha \rightarrow \alpha]$ would give use $\alpha \sim \alpha \rightarrow \alpha$, which would yield an incomputable infinite type.

Principal typings will be very important later when we introduce more advanced type systems as the principal type of $e$ is the only information about $e$ we have to consider when we see $e$ again, given it has no free variables.

\subsubsection{Type Instantiation}

We will call a type $\tau$ an instance of a type $\sigma$ if there is a type substitution $S$ such that transforms $\tau = \tau S$.

\subsection{Type Unification}

If $\tau_1, \tau_2$ are types and $S$ is a substitution, we say that $S$ unifies $\tau_1$ and $\tau_2$ if $\tau_1 S = \tau_2 S$, then we call $S$ their unifier, and we call $\tau_1$ and $\tau_2$ unifiable. \cite{robinson1965machine} (originally for sets of types)

\subsection{Type Checking, Typeability, and Type Inference}

Type checking is the process of validating whether a term can have a given type, whereas typeability is a decision problem of whether the term can be given some type.

Type inference is closely related to typeability but has an additional requirement of giving an actual type for the term, and furthermore, what we usually are interested in is its principal type.

Both type checking and typeability of the simply-typed lambda calculus are decidable, and its type inference is computable \cite{barendregt1992lambda}.

One thing that is a problem for the simply-typed lambda calculus is that it is not Turing-complete. It is strongly normalizing \cite{barendregt1992lambda}, and thus assuming it being Turing-complete, the type inference would have to solve the halting problem. This problem will be answered by the following type systems.

\subsection{Writing Conventions for types}

We will follow the convention of right-associativity of the $\rightarrow$

\section{System F and Parametric Polymorphism}

System F, polymorphic lambda calculus, or second-order lambda calculus introduces parametric polymorphism into the lambda calculus.

Here we can reuse the example of $id = \lambda x . x$ which, given the simply-typed lambda calculus, can have many typings ($id : \tau \rightarrow \tau$ for any $\tau$) which in this type system we can state by one typing:

$$id : \forall \alpha . \alpha \rightarrow \alpha$$

We call such typings (beginning with the $\forall$ quantifier) \textbf{polymorphic} (opposed by \textbf{monomorphic}).

The definition of types extends the definition \ref{defn:typeSTLC} for the simply-typed lambda calculus.

\subsection{Types}

\begin{defn}[Type]
    $$\tau ::= \alpha\ |\ \tau \rightarrow \tau'\ |\ \forall \alpha . \tau$$
    where $\tau, \tau'$ are types and $\alpha$ is a type variable; we can also add type constants.
\end{defn}

In the last case ($\forall \alpha . \tau$) we call $\alpha$ (its occurrence of in $\tau$) a \textbf{bound} type variable. If the variable is not bound we call it \textbf{free}.

\subsection{Inference Rules}

\begin{defn}[Inference rules]
    System F uses the rules of the simply-typed lambda calculus, plus the following:
    $$\boxed{\begin{array}{l c}
        \infer{\Gamma \vdash e : \sigma [\alpha := \tau]}{\Gamma \vdash e : \forall \alpha . \sigma} & (\text{$\forall$-elimination})\\\\
        \infer{\Gamma \vdash e : \forall \alpha . \sigma}{\Gamma \vdash e : \sigma} & (\text{$\forall$-introduction})
    \end{array}}$$
    In $(\text{$\forall$-introduction})$ the type variable $\alpha$ is not free in any any assumptions on which the premise $e : \sigma$ depends.
\end{defn}

In system F the type checking is undecidable, and thus any general type inference algorithm is impossible.

The Hindley-Milner type system is a restriction of it that allows for a simple type inference algorithm on which we will base ours.

\subsection{Type Substitution}

The definition for type substitution and type instances are the same as those for the simply-typed lambda calculus, with the exception that the rewriting does not apply to bound type variables.

\section{Hindley-Milner Type System}

The Hindley-Milner type system, we will refer to it simply as HM, is a fairly minimalistic extension of lambda calculus, or at least syntactically. Semantically it could be seen rather as a restriction of the system F that answers the undecidability of it. For that, see the subsection \emph{Type Schemes}.

Expressions in HM have the form:

\begin{defn}[HM expressions]
    \label{defn:syntaxHM}
    $$e ::= x\ |\ e e'\ |\ \lambda x . e\ |\ \text{let } x = e \text{ in } e'$$
    where $x$ stands for a variable, and $e, e'$ for expressions
\end{defn}

We skipped two more cases that are not necessary for the type inference. We will describe them in the definition \ref{defn:ifElseHM}, and later we will define them as regular functions.

It differs from the syntax of lambda calculus (and that of the system F) only in the addition of the third clause, called the \emph{let statement}. The meaning of this clause is the same as of $e' [x := e]$, but it allows $e$ to be polymorphic, which is otherwise for subexpressions forbidden in HM. A very similar construct $(\lambda x . e') e$ (using abstraction instead of \emph{let}) would result in all occurrences of $x$ in $e'$ having the same monomorphic type in this system. We will later describe definitions for typings and inference rules that will result in this behavior.

There are two additional expressions completing the language that have no significance to the type system and thus we will not further consider them in any of the typing rules, and in the inference algorithm. They are defined as follows:

\begin{defn}[If Then Else and Fix]
    \label{defn:ifElseHM}
    $$e ::= \cdots\ |\ \text{fix } x . e\ |\ \text{if } e \text{ then } e' \text{ else } e''$$
    where $x$ stands again for a variable, and $e, e', e''$ for some expression; `$\cdots$' are to denote that this extends the definition \ref{defn:syntaxHM}.
\end{defn}

The latter clause with the expected meaning, and the former being defined as the least fixed point of $\lambda x . expression$. In the context of type inference the two clauses can be seen as any other application using the following typings (type schemes):

$$\text{fix}: \forall \alpha . \alpha \rightarrow \alpha \rightarrow \alpha$$
$$\text{if}: \forall \alpha . \text{ bool } \rightarrow \alpha \rightarrow \alpha \rightarrow \alpha$$

\subsection{Types}

Types of the HM type system are restricted in that they cannot include quantification of type variables introduced in the F system:

\begin{defn}[Type]
    $$\tau ::= \alpha\ |\ \iota\ |\ \tau \rightarrow \tau'$$
    where $\tau, \tau'$ stand for types, $\alpha$ for a type variable, $\iota$ for a primitive type (type constant), and $sigma$ for a type scheme.
\end{defn}

Compare this definition with the definition \ref{defn:typeSTLC}.

\subsection{Type Schemes}

Type schemes are defined similarly to types in the system F. This restriction means just that the type quantification, if there is any, is always at the very front of the type scheme.

\begin{defn}[Type Scheme]
    \label{defn:schemeHM}
    $$\sigma ::= \tau\ |\ \forall \alpha . \tau$$
    where the former case is a monomorphic type scheme and the latter is polymorphic.
\end{defn}

\subsubsection{Typing extensions}
\label{sssec:typingExt}

In the HM type system, it is common to introduce the following type operators: cartesian products (tuples): $\times$; disjoint sums (tagged unions): $+$; and lists. Existence of these is not required for the type system, and so we will avoid them going forward. Their reintroduction is trivial as they can be described as regular type constructors.

\subsection{Typing}

Typing of expressions is modified by the introduction of type schemes into assigning a type scheme to a variable instead of assigning a type. See the definitions \ref{defn:typingSTLC} and \ref{defn:schemeHM}.

\subsection{Type Instantiation}

Type definition of type instantiation carries over from the system F and the simply-typed lambda calculus, but it also applies to \emph{type schemes}. \cite{damas1982principal}

A notable change being an addition of an \textbf{generic instance} of a type scheme $\sigma = \forall \alpha_1, \alpha_2 \cdots \alpha_n . \tau$ which is defined as $\sigma' = \forall \beta_1, \beta_2 \cdots \beta_m . \tau'$ such that $\tau' = \tau S$ for some type substitution $S = [\alpha_1 := \tau_1, \cdots \alpha_n := \tau_n]$, where type variables $\beta_i$ are not free in $\sigma$. We then write $\sigma > \sigma'$.

Note that $\sigma > \sigma'$ implies $\sigma S > \sigma' S$, however it does not imply $\sigma S > \sigma$.

\subsection{Inference Rules}

\begin{defn}[Inference rules]
    $$\boxed{\begin{array}{l c}
        \infer[(x : \sigma \in \Gamma)]{\Gamma \vdash x : \sigma}{} & (\text{var}) \\\\
        \infer{\Gamma \vdash e e' : \tau}{\Gamma \vdash e : \tau' \rightarrow \tau \quad \Gamma \vdash e' : \tau'} & (\text{application}) \\\\
        \infer{\Gamma \vdash \lambda x . e : \tau' \rightarrow \tau}{\Gamma_x \cup \{x : \tau'\} \vdash e : \tau} & (\text{abstraction}) \\\\
        \infer[(\sigma > \sigma')]{\Gamma \vdash e : \sigma'}{\Gamma \vdash e : \sigma} & (\text{instantiation}) \\\\
        \infer[(\alpha \text{ not free in }\Gamma)]{\Gamma \vdash e : \forall \alpha . \sigma}{\Gamma \vdash e : \sigma} & (\text{generalization}) \\\\
        \infer{\Gamma \vdash \text{let } x = e \text{ in } e' : \tau}{\Gamma \vdash e : \sigma \quad \Gamma_x \cup \{x :\sigma\} \vdash e' : \tau} & (\text{let polymorphism})
    \end{array}}$$
    Here $\Gamma$ stands for the context, $\sigma$ stands for a type scheme, $\tau, \tau'$ for some types, $\alpha$ for a type variable, $x$ for a regular variable, and $e, e'$ for expressions. $\Gamma_x$ stands for the context $\Gamma$ with any assumption about $x$ removed.
\end{defn}

The \emph{instantiation} rule replaces the $\forall$-elimination rule of the system F. Apart from that, the \emph{let polymorphism} rule is the only real extension of the rules, balancing the type restriction introduced before.

\subsection{Type Inference}

We will introduce the algorithm W \ref{w}. It produces the principal type scheme for the given program (expression).

\subsubsection{Most General Unification}

\begin{algorithm}[t]
\caption{Unification Algorithm \cite{robinson1965machine}}
\label{mgu}
\begin{algorithmic}[1]
\Function{$mgu$}{$\tau, \tau'$}
    \State $S_0 \gets \epsilon$
    \For{$k = 0$ to $\infty$}
    \If{$\tau S_k = \tau' S_k$}
        \Return $S_k$
    \EndIf
    \State Let $\tau_k, \tau'_k$ be leftmost corresponding well-formed sub-expressions in which $\tau$ and $\tau'$ differ \Comment if the expressions are represented by syntactic trees, $\tau_k$ and $\tau'_k$ have the same relative path from their respective roots - then leftmost means having the lowest inorder rank).
    \If{$\tau_k$ is a variable not occurring in $\tau'_k$}
        \State $S_{k+1} \gets S_k [\tau_k := \tau'_k]$
    \ElsIf{$\tau'_k$ is a variable not occurring in $\tau_k$}
        \State $S_{k+1} \gets S_k [\tau'_k := \tau_k]$
    \Else
        \State Terminate with an \textit{occurrence check} error
    \EndIf
    \EndFor
\EndFunction
\end{algorithmic}
\end{algorithm}

One of the main parts of the type inference algorithm is the type unification, and since we are usually after principal types, the most general unification.

The most general unifier of types $\tau, \tau'$ is a unifier $S$ such that for every other unifier $U$ of $\tau$ and $\tau'$ there exists a substitution $R$ such that $U = S R$. \cite{damas1982principal}

We will further require the algorithm to use only variables from $\tau$ and $\tau'$.

We present an algorithm \ref{mgu}, which is a variation of Robinson's unification algorithm. We will use a modification of this algorithm even in the next type system.

\subsubsection{The Type Assignment Algorithm W}

\begin{algorithm}[t]
\caption{The algorithm W \cite{milner1978theory}}
\label{w}
\begin{algorithmic}[1]
\Function{W}{$\Gamma, e$}
    \If{$e$ is a variable $\wedge\ e : \forall \overline{\alpha} . \tau' \in \Gamma$}
    \State let $\overline{\beta}$ be a list of \emph{new} type variables s.t. $|\overline{\beta}| = |\overline{\alpha}|$
    \State $(S, \tau) \gets (\epsilon, \tau' [\overline{\alpha} := \overline{\beta}])$
    \ElsIf{$e$ \textbf{matches} $e_1 e_2$}
    \State $(S_1, \tau_1) \gets W(\Gamma, e_1)$
    \State $(S_2, \tau_2) \gets W(\Gamma S_1, e_2)$
    \State let $\beta$ be a \emph{new} type variable
    \State $S_3 \gets mgu (\tau_1 S_2, \tau_2 \rightarrow \beta)$
    \State $(S, \tau) \gets (S_1 S_2 S_3, \beta S_3)$
    \ElsIf{$e$ \textbf{matches} $\lambda x . e_1$}
    \State let $\beta$ be \emph{new} type variable
    \State $(S, \tau_1) \gets W(\Gamma_x \cup \{x : \beta\}, e_1)$
    \State $\tau \gets (\beta S \rightarrow \tau_1)$
    \ElsIf{$e$ \textbf{matches} $\text{let } x = e_1 \text{ in } e_2$}
    \State $(S_1, \tau_1) \gets W(\Gamma, e_1)$
    \State $(S_2, \tau) \gets W(\Gamma_x S_1 \cup \{close_{\Gamma S_1}(\tau_1)\}, e_2)$
    \State $S = S_1 S_2$
    \Else
    \State Terminate with an error stating that $e$ is not typeable
    \EndIf
    \State \Return $(S, \tau)$
\EndFunction
\end{algorithmic}
\end{algorithm}

Before we describe the algorithm itself, we first need to define the closure of a type $\tau$ under assumptions $\Gamma$:

\begin{defn}[$close_\Gamma(\tau)$]
    \label{defn:close}
    $$close_\Gamma(\tau) ::= \forall \alpha_1, \dots \alpha_n . \tau$$
    where $\{\alpha_1, \dots \alpha_n\} := free(\tau) \backslash free(\Gamma)$, $free$ stands for the set of free variables, see the definition \ref{defn:boundFree}.
\end{defn}

The algorithm \ref{w} is a function $W$ which takes a pair $(\Gamma, e)$ where $\Gamma$ is a set of assumptions and $e$ is an expression in the context of $\Gamma$. Then $W$ returns a pair $(S, \tau)$ such that:

$\Gamma S \vdash e : \tau$ and furthermore $close_{\Gamma S}(\tau)$ is a principal type scheme of $e$ under $\Gamma S$.

\subsection{Extensions of the Type System}

Pure HM allows for just parametric polymorphism. However, without any ad-hoc polymorphism or subtyping, this means there is only one principal type for every symbol in the program and just one definition. This is useful in cases where an algorithm (or a function) works on parameters of arbitrary types. A good example of that could be the $K$ combinator (sometimes called `const function'). We call these algorithms generic.

The problem is some functions work only on a particular type, for example, basic arithmetic functions. We would like to be able to say that it makes sense to add two integers, or two real numbers, but not adding two boolean values.

A trivial extension would be the introduction of tagged unions, see \emph{Typing extensions} in \ref{sssec:typingExt}. But if we were to define the $+$ operator as $(+) : Int + Real \rightarrow Int + Real \rightarrow Int + Real$ we solve the problem with booleans, one different problem comes to the surface: we would like to say that adding two integers yields an integer whereas adding two real numbers yields a real number, one more argument could also be that we would like to forbid adding integers to real numbers (maybe we want to require all data conversions be explicit).

So what we would like is some way of expressing that $(+)$ takes two parameters of a type we can perform addition on and that the function yields us a result of the same type.

This can be answered by introducing \textbf{type classes}.

\subsubsection{Type Classes and their Methods and Instances}

The idea of type classes (parametrized by generic types) is that we put constraints on types of some variables (or functions) we use in our algorithms. Those constraints tell us that the usage of these algorithms is valid only in the context where the constraints are satisfied (this allows us to implement otherwise impossible generic algorithms, for example, sorting).

And also, the functions defined inside a type class (we will call them \textbf{methods}) can have a specialized implementation for each \emph{instance}, this was actually the main motivation for their creation: $Eq$ class with a method $==$ which needs to be specialized for each type. \cite{hall1994type}

\textbf{Instances} of type classes (parametrized by a type instance of the type of the corresponding type class) are what allows for ad-hoc polymorphism in a very well controlled manner \cite{wadler1989make}, they are a limited (but powerful) version of classic overloading (we will describe overloading shortly). It differs from classic overloading in that all instances have to share a common supertype and they are not allowed to overlap (share a common type instance - this can be proven by the unification algorithm) - this "no overlapping" rule will be important for instantiation (generating code with concrete monotypes: overlapping would mean having, for the given parameters, two implementations simultaneously).

Type classes can be added to the algorithm $W$ defined above with only little changes, for example, by simplifying thih \cite{jones1999typing}.

\subsubsection{Subtyping and Overloading}

Subtyping and overloading are slightly different approaches to the same problem we discussed with specifying what types the $(+)$ function can act on.

If we want to use either of those, we have to constrain them somehow, or they make type inference undecidable. We will consider the variant defined by professor Geoffrey S. Smith \cite{smith1993polymorphic}.

\textbf{Subtyping}, however, does not solve that problem very well as if we specify that $Int$ is a subtype of $Real$, then we have to accept that $(+)$ applied to two real operands can still return an integer. That does not seem that bad from the mathematical point of view, but it can lead to implicit data conversions, which can be undesirable, especially if we were to use the type system in the context of low-level programming.

\textbf{Overloading}, giving a symbol more definitions, then is not only undecidable in the context of parametric polymorphism of HM but it also completely contradicts the ideas of parametric polymorphism, where a function's type should describe the function and its effect on the program (in languages like Haskell or ML, the type also describes data dependencies). \cite{palsberg2012overloading}

\section{Overloading with Type Classes}

Type classes, as stated before, are good for stating an algorithm makes sense only for some types or that it can have a different implementation for each type, usually to enhance performance by writing specialized algorithms for common or critical types instead of writing a generic slow algorithm.

The usefulness of classes can be shown on `Ord'. This constraint requires that two variables of type $\alpha$ s.t. $Ord(\alpha)$, we can say `$\alpha$ is (in) Ord'. But not everything can be partially ordered, take directed graphs, for example, where the reachability of one vertex from another is in `Ord' only if the graph is a directed acyclic graph, in which case we can use algorithms expecting `Ord' on problems concerning reachability, but not otherwise.

`Ord' can also show how we could want different instances for different types. Let's say `Ord' has one method $<=$ of type $a \rightarrow a \rightarrow Bool$ with the expected behavior.

For two `Int's we just compare them and return whether the first is lower than the other, but if we have, let's say `Customer's, with different Id numbers as keys in some map (or dictionary), then we do not need to compare whole customers, but just their Ids.

\section{Haskell Type System}

For our purposes, we will (limit) the Haskell language significantly, ignoring any pattern-matching features and almost all syntax sugar, the only important feature for us is the type system. The syntax is that of HM but extended by the possibility of giving explicit typings (with type schemes, see the definition \ref{defn:typeSchemes}) and defining/using type constructors and type classes. See thih \cite{jones1999typing} for an actual implementation we will use in the CHM compiler.

\subsection{Type Constructors}

Haskell generalizes the HM type system and the HM syntax mostly via the notion of type constructors. If they take no type arguments, we call them nullary. Those are the ones we already call \emph{types} in the HM type system. We call a type constructor n-ary when it takes exactly $n$ type arguments to form a (nullary) value type. For example, the type constructor $(\rightarrow)$ is binary, and it constructs the type of a function.

We can formally define type constructors similarly to functions.

\subsection{Kinds}

The next thing the Haskell type system adds to the HM type system is the notion of kinds. Kinds are used for the classification of type constructors and for checking the validity of their usage in the program.

Kinds are related to types similarly to how types are related to functions and variables, and they follow their syntactic rules but with type constructors instead of values. The most simple kind $*$ (star) represents all types capable of having a value: all legal types from the HM type system would fall into this category.

Then we have kinds of a form $k_1 \rightarrow k_2$ which take a type of a kind $k_1$ and return a type of kind $k_2$. For example a $(\rightarrow)$ type constructor has a kind $* \rightarrow (* \rightarrow *)$, usually written with considering right associativity as $* \rightarrow * \rightarrow *$. This means $(\rightarrow)$ takes two types and returns a new type - this reflects the fact that a type of a function is defined by its parameter's type and its return type.

Kinds are less complex then types as they are defined recursively only from the two rules defined in the previous two paragraphs.

For example, we can define a very common type constructor $List: * \rightarrow *$ which takes a type $a$ and constructs a type representing a linked list of values of type $a$. We can use this type constructor to show the reason behind the notion of kinds as it would not make much sense for a value to have a type just $List$ without it being applied to any type: "a list of what?"

We will formally define kinds as:

\begin{defn}[kinds]
    $$k ::= *\ |\ k' \rightarrow k''$$
    where $k, k', k''$ are kinds.
\end{defn}

\subsection{Types}

We will define types formally as:

\begin{defn}[Type]
    $$\tau ::= x\ |\ c : k\ |\ \tau \tau'$$
    where $\tau, \tau'$ are types, $x$ is a type variable, $c$ is a type constant, and $k$ is a kind. We will, for example consider the $(\rightarrow)$ type constructor a regular type of the kind $* \rightarrow * \rightarrow *$.
\end{defn}

\subsection{Type Schemes}

Type schemes will be defined similarly to those of the HM type system, differing only in what we consider a type and in that there can be constraints that the types have to satisfy. It should also be noted that only types of the $*$ kind can be assigned to values.

\begin{defn}[Type Scheme]
    \label{defn:typeSchemes}
    $$\sigma ::= \forall \alpha_1, \dots \alpha_n . qt\ |\ qt$$
    where $qt$ stands for a qualified type (a constrained type) and $\alpha_i$ stand for distinct type variables
\end{defn}

\subsubsection{Qualified Types and Predicates}

A qualified type is a type with an added set of predicates every instance of it has to satisfy.

\begin{defn}[Qualified Types and Predicates]
    $$qt ::= \{\beta_1 \in C_1, \beta_2 \in C_2, \dots \beta_m \in C_m\} \Rightarrow \tau\ |\ \tau$$
    where $\beta_j$ name type variables such that $\{\beta_1, \dots \beta_m\} \subseteq \{\alpha_1, \dots \alpha_n\}$ ($\alpha_1 \dots \alpha_n$ from the definition \ref{defn:typeSchemes}, qualified types occur always in the context of type schemes), $C_j$ names a type class, and $\tau$ names a type. We call the pair $\beta_j \in C_j$ a \textbf{predicate (constraint)}
\end{defn}

Going forward, we will allow the following forms for type schemes $\forall \{\}. qt = qt$ and for qualified types $\{\} \Rightarrow \tau = \tau$ to simplify other definitions so we do not need to consider separate cases.

\subsection{Type substitution}

With the introduction of qualified types, we should explicitly state that the type substitution applies to qualified types, just like if the type parameters of the predicates were a part of the type itself.

\subsection{Type Inference Rules}

The type inference rules for the Haskell type system are almost the same as the type inference rules for the HM type system, but we have to add rules for the inference of the type class constraints:

\begin{defn}[Inference rules]
    $$\boxed{\begin{array}{l c}
        \infer[(x : \sigma \in \Gamma)]{\Gamma \vdash x : \sigma}{} & (\text{var}) \\\\
        \infer{\Gamma \vdash e e' : \mathbf{P} \cup \mathbf{P'}  \Rightarrow \tau}{\Gamma \vdash e : \mathbf{P} \Rightarrow \tau' \rightarrow \tau & \Gamma \vdash e' : \mathbf{P'} \Rightarrow \tau'} & (\text{application}) \\\\
        \infer{\Gamma \vdash \lambda x . e : \mathbf{P} \cup \mathbf{P'} \Rightarrow \tau' \rightarrow \tau}{\Gamma_x \cup \{x : \mathbf{P'} \Rightarrow \tau'\} \vdash e : \mathbf{P} \Rightarrow \tau} & (\text{abstraction}) \\\\
        \infer[(\sigma > \sigma')]{\Gamma \vdash e : \sigma'}{\Gamma \vdash e : \sigma} & (\text{instantiation}) \\\\
        \infer[(\alpha \text{ not free in }\Gamma)]{\Gamma \vdash e : \forall \alpha . \sigma}{\Gamma \vdash e : \sigma} & (\text{generalization}) \\\\
        \infer{\Gamma \vdash \text{let } x = e \text{ in } e' : \mathbf{P} \cup \mathbf{P'} \Rightarrow \tau'}{\Gamma \vdash e : \forall \bar{\alpha} .  \mathbf{P} \Rightarrow \tau & \Gamma_x \cup \{x :\forall \bar{\alpha} .  \mathbf{P} \Rightarrow \tau \} \vdash e' : \mathbf{P'} \Rightarrow \tau'} & (\text{let polymorphism})
    \end{array}}$$
    where $\tau, \tau'$ stand for types, $e, e'$ stand for expressions, $x$ stands for a type variable, $\bar{\alpha}$ stands for a list of type variables, $\Gamma$ stands for the set of assumptions, $\Gamma_x$ stands for the set of assumptions without any judgement of x, and $\mathbf{P}, \mathbf{P'}$ stand for type predicates.
\end{defn}

Those rules are the same as those of the HM type system with only a trivial difference that complex expressions retain predicates of all of their subexpressions.


\subsection{Type Instances and Type Unification Algorithm}

The addition of type constructors and type classes does not require any change to the definition of the type unification algorithm. The only notable fact is that unification can succeed only if the unified types share kinds.

As for type instances, the definition is without any change.

\subsection{The Type Inference Algorithm}

The type inference algorithm has to account for the bottom-up propagation of predicates that the inferred types have to satisfy. The basis for the inference algorithm is the algorithm W \ref{w}. For the definition, see thih \cite{jones1999typing}. In thih \lstinline[language=haskell]{tiExpr} is the part of type inference that encompasses the extension of the algorithm W. The definition in thih differs slightly from what we have discussed in this chapter in that it does not consider just the syntax from the definition \ref{defn:syntaxHM} with the extensions we presented, but it also considers some features of the Haskell language (for example pattern matching).
