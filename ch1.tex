\chapter{Functional programming with types}

\xxx{O cem je funkcionalni programovani (vypocet je modelovanej substituci, diky tomu se v tom substitucni veci delaj skvele)}

\section{Simply-typed lambda calculus}

Describing lambda calculus is important in both the context of type inference and in the context of (functional)
programming as a whole.

In this thesis we will use lambda calculus as described in Introduction to Lambda Calculus.  % TODO: insert something

Lambda calculus is formally described as:

$
    variable ::= 'v' | variable ''' \\
    \lambda-term ::= variable | '(' \lambda-term \lambda-term ')' |  '(\lambda' variable \lambda-term ')'
$

It is a simple language in which we can formally describe any computational problem.

Choosing lambda calculus, or its derivatives might seem counterintuitive at first, because it
is not very efficient in expressing imperative ideas like mutability for example, but for
purposes of type inference we don't need to know the actual meaning of the code (like for example
if a loop stops), we can just limit ourselves to modelling type dependencies in the program and
everything on top of that is just for human readability, debugging purposes, demonstration and
primarily for possible extension.

Lambda calculus can quite easily model any C program, but one has to be careful at distinguishing
between initialization and assignment as initialization creates binding (in lambda calculus expressed
as substitution for a bound variable in an abstraction) and assignment is a function that takes two
expressions of the same type and returns the first (here we demonstrate that we don't need to
model the whole behavior of the program, because copying of the value doesn't have any effect on
the types of the arguments).

Here we will use $\lambda_C$ to denote transformation from C to lambda calculus,
this transformation can be actually done in a multiple of ways, but in this thesis we will use one that tries to
reflect the structure of the original code as closely as possible.

Models of some C constructs can be as follows (grammar taken from ): % TODO: http://www.quut.com/c/ANSI-C-grammar-y.html

$
    \lambda_C\left(additive\_expression '+' multiplicative\_expression\right) = \\
    plus_C (\lambda_C additive\_expression) (\lambda_C multiplicative\_expression)
$

This example demonstrates the most simple case where  where we can model the C construct "one to one",
but there are more tricky examples like the following one:

$
    \lambda_C\left(declaration\_specifiers IDENTIFIER '=' initializer\right) = \\
    (\lambda IDENTIFIER : \lambda_C(declaration\_specifiers) .\ \dots) (\lambda_C initializer)
$

Initialization creates binding (as stated before) so we have to model that by creating a new abstraction and put the whole
part of the function's body that follows inside this abstraction (this in effect means that all the return values of these nested functions
and of the function which contains them will share the same type), a similar thing are switch statements where the switch "call" can be modeled
as an initialization to an anonymous variable and the single case statements as assignments to this variable.


\section{Hindley-Milner polymorphism}

(As described in PTSfPwOaS).
Pure HM allows for just parametric polymorphism without any ad-hoc polymorphism
or subtyping, this means there is one principal type for every symbol in the program and it has just one definition, this is useful in cases
where an algorithm works on any types of its algorithms, good example of that would be swapping the values of two pointers (this algorithm cannot
take advantage of knowing the exact type unless we take into consideration some really nontraditional architectures).

We will then break this one
rule by introducing type classes and type families, two ideas with the same basis where we put constraints on types and we can
create non-overlapping instances for them, or we can put these constraints on a regular symbol to denote that the algorithm either works only on
some types or doesn't make sense on others, this is especially useful in static analysis of the code.

Other solutions of this are overloading and subtyping, but the type inference of the most common form of overloading is undecidable
as described by, so overloading wouldn't be a good solution in our case. % TODO: NP overload

It also contradicts the idea of parametric polymorphism used in this thesis, where every implementation of an function can be
described as a proof of a special case of a theorem described by the principal type of the function and this type is therefore
valid for all its implementations.

Subtyping % TODO: finish this

\section{Overloading with type classes}

Type classes, as stated above, are good for stating an algorithm makes sense only for some types or that it requires a different implementation for
different types, usually to enhance performance by writing specialized algorithms for common or critical types instead of writing a general slow algorithm,
this was the main motivation behind developing those, % TODO
viz different physical instructions for comparing common numeric values.

Usefulness of classes can be shown on `Ord', this constraints requires that two variables of type $\alpha$ s.t. $Ord(\alpha)$, we can say `$\alpha$ is (in) Ord'.
But not everything can be partially ordered, take directed graphs, for example, where the reachability of one vertex from another is in `Ord' only
if the graph is a directed acyclic graph, in which case we can use algorithms expecting `Ord' on problems concerning reachability, but not otherwise.

`Ord' can also show how we could want different instances for different types.
Let's say `Ord' has one method $<=$ of type $a -> a -> Bool$ with the expected behavior. % TODO: explain a method

For two `Int's we just compare them and return whether the first is lower than the other,
but if we have, let's say `Customer's, with different Id numbers as keys in some map (or dictionary) then we don't need to compare whole customers,
but just their Ids.

method: a function which is defined inside a class and is expected to be instantiated alongside the class. % TODO

\xxx{zhruba popsat jak to vysvihnul \citet{jones1999typing}.}

\xxx{test normalni citace: \cite{jones1999typing}.}

\section{Haskell type system}

\xxx{Teoretickej skok do budoucnosti --- mirnej review o tom jak se to ve skutecnosti dela v haskelu. Treba ukazat ze MPTCs jsou nerozhodnutelny protoze umej simulovat prolog.}
