\chapter{The STL Library}
\label{stl}

In every language, there are many functions and patterns used in almost every program. We can expect the same happening to the CHM language if it becomes popular.

We present a demonstrative \emph{STL} library implementation, which should give an example of how some functions provided by it might look. This library is implicitly linked with every program.

The \emph{STL} library is split into two files \lstinline[language=sh]{stl.h},which can be included to CHM sources, and \lstinline[language=sh]{stl.c}, which implements some IO functions currently unimplementable in the CHM language.

\section{Functions Facilitating Memory Management}

Here we will present functions \lstinline{new} and \lstinline{delete} for polymorphic memory allocation and disposal, respectively; and a function \lstinline{nullptr} which serves as polymorphic \lstinline{NULL} pointer.

The following code is taken from \lstinline[language=sh]{stl.h}:

\begin{lstlisting}
extern void *malloc(size_t);
extern void free(void *ptr);

<a>
a *nullptr()
{
    return (a *)NULL;
}

<a>
a *new()
{
    return (a *)malloc(sizeof(a));
}

<a>
void delete(a *where)
{
    return free((void *)where);
}
\end{lstlisting}

\section{Functions Facilitating IO}

Here we show how the \emph{STL} library could implement a generic \lstinline{print} function. It is a method of the \lstinline{Print} type class, as every type requires a specific implementation of how it should be printed.

The implementation shown here takes the object that is to be printed by a pointer, which should not matter for trivial types, as the functions will likely be inlined by GCC, but for larger objects (requiring a nontrivial implementation of printing), passing the object by value might cause unwanted overhead.

The following code is again taken from \lstinline[language=sh]{stl.h}:

\begin{lstlisting}
// the following two functions implemented in stl.c
extern void __print_char(char);
extern void __print_int(int);

<a>
class Print {
    void print(a *);
}

instance Print<char> {
    void print(char *a)
    {
        __print_char(*a);
    }
}

instance Print<int> {
    void print(int *a)
    {
        __print_int(*a);
    }
}
\end{lstlisting}
