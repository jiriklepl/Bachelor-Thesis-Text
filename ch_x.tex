\section{CHM specification}

\subsection{CHM headers}

Polymorphic functions and structures are constructed using special annotations (called CHM headers) following grammar:

chm_head := '<' new_typevar_list '>' | '<' new_typevar_list ':' constraint_list '>'| '<' constraint_list '>'

new_typevar_list := new_typevar | new_typevar_list ',' new_typevar
new_typevar := IDENT

constraint_list := constraint | constraint_list ',' constraint
constraint := unification | predicate

unification := IDENT '~' type

predicate := CLASS '<' type_list '>'
type_list := DECLARATION | type_list ',' DECLARATION

This is a simplified version of a piece of the actual grammar used in the implementation,
where we used all-caps for tokens that are not local to the scope of chm heads: `IDENT' means a new variable,
`CLASS' is a name of a type-class, % TODO: link
`DECLARATION' is as described in the C standard.

token new_typevar syntactically behaves as defining a new typedef.

A unification provides a way of creating a type alias akin to typedefs (syntactically equal) with the benefit of being local to the definition
it precedes, but without creating an actual typedef which would interfere with any following code.

\subsection{Classes}

The other addition CHM brings to C are classes.

They are defined (simplified version) as follows:

class := 'class' IDENT chm_head '{' EXT_DECLARATIONS '}'
instance := 'instance' IDENT '<' type_list '>' '{' EXT_DECLARATIONS '}'


