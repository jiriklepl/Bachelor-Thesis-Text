\chapter{Using the CHM Compiler}

In this appendix, we will discuss how to use the demonstration implementation of the CHM compiler.

\section{Getting the Compiler}

Its source code can be viewed and downloaded via Github \cite{jiriklepl2020chmcompiler}.

Note that the compiler requires all its submodules to be up-to-date. That can be achieved using the following command to clone the repository:

\begin{lstlisting}[language=sh]
git clone --recursive \
	https://github.com/jiriklepl/Bachelor-Thesis
\end{lstlisting}

\section{Build Dependencies}

The main dependencies of the compiler are \emph{ghc} (at least version 8.10) and \emph{cabal-install} (at least version 3.2). It depends on many packages automatically acquirable via \emph{cabal-install} which can be found in each sub-project's \lstinline{.cabal} file.

\section{Building}

The package can be built with either of the following commands:

\begin{lstlisting}[language=sh]
    make build

    # or
    cabal new-build all
\end{lstlisting}


\section{Usage}

The compiler can be run by either of the following commands (note that \lstinline[language=sh]{<arguments>} stands for the arguments passed to the compiler):

\begin{lstlisting}[language=sh]
    ./run.sh <arguments>

    # or
    cabal new-run CHM-main <arguments>
\end{lstlisting}

The first one might require adjusting the read and execution permissions on the \lstinline[language=sh]{run.sh} file.

The implementation is for testing and development of the language and thus the directory it is run in has to contain the files \lstinline[language=sh]{stl.c} and  \lstinline[language=sh]{stl.h}, which implement the \emph{stl} library implicitly linked with every program.
